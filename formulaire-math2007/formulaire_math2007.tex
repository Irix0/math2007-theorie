\documentclass[a4paper]{article}
\usepackage[utf8]{inputenc}
\usepackage{array}
\usepackage{amsthm}
\usepackage{amsmath}
\usepackage{amssymb}
\usepackage{graphicx}
\usepackage{subcaption}
\usepackage[french]{babel}

\newtheorem{prop}{Propriété}[section]
\newtheorem{definition}{Définition}[section]
\title{MATH2007}
\author{louan20 }
\date{December 2021}

\begin{document}

\begin{center}
    \huge\fbox{\textbf{MATH2007}} \\
    \normalfont
\end{center}
\part{Formulaire}
\large
\section{Dérivées et primitives directes}

\renewcommand{\arraystretch}{1.5}

\begin{equation*}
\begin{array}{|l|l|l|}
  \hline
  f(x) & Df(x) & F(x) \\
  \hline\hline
  C & 0 & Cx \\
  \hline
  x & 1 & \frac{x^2}{2} \\
  \hline
  x^n & nx^{n-1} & \frac{x^{n+1}}{n+1} \\
  \hline
  (x-a)^n & n(x+a)^{n-1} & \frac{(x+a)^{n+1}}{n+1} \\
  \hline
  \frac{1}{(x+a)^n} & -\frac{n}{(x+a)^{n+1}} & \\
  \hline
  \frac{1}{x+a} & & ln(|x+a|)\\
  \hline
  ln(x+a) & \frac{1}{x+a} & \\
  \hline
  e^x & e^x & e^x \\
  \hline
  sin(x) & cos(x) & -cos(x) \\
  \hline
  cos(x) & -sin(x) & sin(x) \\
  \hline
  tan(x) & \frac{1}{cos^2(x)} & \\
  \hline
  cotg(x) & \frac{-1}{sin^2(x)} & \\
  \hline
  arcsin(x) & \frac{1}{\sqrt{1-x^2}} & \\
  \hline
  arcos(x) & -\frac{1}{sqrt{1-x^2}} & \\
  \hline
  arctan(x) & \frac{1}{1+x^2} & \\
  \hline
  arctan(x) & \frac{-1}{1+x^2} & \\
  \hline

 \end{array}
\end{equation*}
 \\
\pagebreak
 \section{Calcul vectoriel : notions analytiques}
 \begin{prop}[Produit scalaire]
  \begin{equation*}
    \vec{u} \bullet \vec{v} = u_x.v_x + u_y.v_y
  \end{equation*}
  \end{prop}

  \begin{prop}[Projection orthogonale]
    \begin{equation*}
      \vec{u'} = \frac{\vec{u} \bullet \vec{v}}{||\vec{v}||^2}\vec{v}
    \end{equation*}
  \end{prop}

  \begin{prop}[Produit vectoriel]
  \item Voici une méthode pour calculer le produit vectoriel :
    \begin{figure}[!h]
      \begin{subfigure}[b]{60mm}
          \includegraphics[scale=0.4]{produit vectoriel/frame_0_delay-1.5s.jpg}
      \end{subfigure}
      \begin{subfigure}[b]{60mm}
          \includegraphics[scale=0.4]{produit vectoriel/frame_1_delay-1.5s.jpg}
      \end{subfigure}
      \begin{subfigure}[b]{60mm}
          \includegraphics[scale=0.4]{produit vectoriel/frame_2_delay-1.5s.jpg}
      \end{subfigure}
      \begin{subfigure}[b]{60mm}
          \includegraphics[scale=0.4]{produit vectoriel/frame_3_delay-1.5s.jpg}
      \end{subfigure}
    \end{figure}
  \end{prop}
  \pagebreak

  \section{Coniques}
  \subsection{Ellipses}

  \begin{equation*}
    \frac{x^2}{a^2} + \frac{y^2}{b^2} = 1
  \end{equation*}
  \smallskip
  Intersections avec les axes : $(a, 0) ; (0, b)$
  \subsubsection{$a>b$}
  L'excentricité $e = \frac{c}{a}$ \\
  Les foyers F sont $(c,0) ; (-c, 0)$ \\
  où $c = \sqrt{a^2-b^2}$

  \subsubsection{$a<b$}
  L'excentricité $e = \frac{c}{b}$ \\
  Les foyers F sont $(0,c) ; (0, -c)$ \\
  où $c = \sqrt{b^2 - a^2}$

  \subsection{Hyperboles}
  \begin{align}
    \label{hyp:1}
    \frac{x^2}{a^2} - \frac{y^2}{b^2} = 1 \\
    \label{hyp:2}
    \frac{y^2}{b^2} - \frac{x^2}{a^2} = 1
  \end{align}
  \smallskip
  Intersection avec l'axe des x : \eqref{hyp:1}$(a, 0)$ ou \eqref{hyp:2} $(0, a)$  \\
  L'excentricité $e = \eqref{hyp:1}\frac{c}{a}$ ou \eqref{hyp:2}$\frac{c}{b}$\\
  Les foyers F sont \eqref{hyp:1} $(c,0) ; (-c, 0)$ ou \eqref{hyp:2} $(0,c) ; (0, -c)$ \\
  où $c = \sqrt{a^2+ b^2}$
  Les asymptotes sont : $y = \eqref{hyp:1}\pm \frac{b}{a}x$ ou \eqref{hyp:2}$\pm \frac{a}{b}x$\\

  \smallskip

  \subsection{Paraboles}
  \begin{align}
    \label{para:1}
    y = 2px^2 \\
    \label{para:2}
    x = 2py^2
  \end{align}
  L'excentricité est 1 \\
  Le foyer est (p/2, 0) pour \eqref{para:1} ou (0, p/2) pour \eqref{para:2}

\part{Théorie}
\section{Monotonie et extrema}
\begin{prop}[Monotonie et dérivation]
Soit f une fonction réelle dérivable sur I = ]a,b[. On a, f est croissant (resp. décroissant) sur I ssi Df est une fonction positive (resp. négative) sur I.\\
(Si Df l'est strictement sur I, alors f égalemment. (réciproque fausse))
\end{prop}
\begin{prop}[Extrema et dérivation]
  Soit f une fonction dérivable sur I = ]a,b[ et soit $x_0 \in I$.
  \begin{enumerate}
    \item Si $x_0$ est un extremum local de f sur I, alors $Df(x_0)$ = 0.
    \item Lorsque la fonction appartient à $C_2(I)$, pour caractériser un extremum, on peut donner un critère utilisant la valeur
          de la dérivée seconde :
          \begin{itemize}
            \item si $Df(x_0)$ = 0 et $D^2f(x_0) < 0$ alors $x_0$ est un maximum local;
            \item Si $Df(x_0$) = 0 et $D^2f(x_0) > 0$ alors $x_0$ est un minimum local.
          \end{itemize}
  \end{enumerate}
\end{prop}

\section{Primitivation}
\begin{definition}
  Soit f une fonction définie sur un intervalle ouvert I de $\mathbb{R}$. On appelle primitive de f sur I toute fonction dérivable F sur
  I telle que DF(x) = f(x), $x \in I$
\end{definition}
\begin{prop}
  Une fonction continue sur ]a,b[ admet toujours une primitive sur cet intervalle.
\end{prop}
\begin{prop}[Unicité]
  Soient $F_1 et F_2$ deux primitives de f sur ]a,b[ qui vérifient $F_1(x_0) = r = F_2(x_0)$. Comme la dérivée d'une constante est nulle, il existe $c \in \mathbb{R}$ tel que $F_1(x) = F_2(x) + c$ pour tout $x \in$ ]a,b[
\end{prop}

\section{Continuité et dérivabilité}
\begin{definition}[Continuité]
  $x_0$ est un point de continuité pour f si $\displaystyle{\lim_{x \to x_0}f(x)}$ existe.
\end{definition}

\begin{definition}[Dérivabilité]
  f est dérivable en $x_0$ si $\displaystyle{\lim_{x \to x_0}\frac{f(x)-f(x_0)}{x-x_0}}$ existe et est finie.
\end{definition}

\begin{prop}[Lien entre dérivabilité et continuité]
  Soit f une fonction dérivable sur ]a,b[, alors f est continue sur ]a,b[.\\
  \textbf{Preuve}. \\
  Soit f une fonction dérivable sur I = ]a,b[ et soit $x_0 \in I$. Alors pour tout $x \in I$, $x \neq x_0$, on a
  \begin{align*}
    f(x) - f(x_0) &= (x-x_0)\frac{f(x) - f(x_0)}{x-x_0}\\
    \text{comme} &\displaystyle{\lim_{x \to x_0}} \frac{f(x) - f(x_0)}{x-x_0} = Df(x_0) \\
    \text{on obtient} &\displaystyle{\lim_{x \to x_0}} f(x) - f(x_0) \\
    &=\displaystyle{\lim_{x \to x_0}}(x-x_0) \displaystyle{\lim_{x \to x_0}}\frac{f(x) - f(x_0)}{x-x_0} \\
    &= 0.Df(x_0) = 0
  \end{align*}
\end{prop}
\end{document}
